\documentclass{article}
\usepackage[utf8]{inputenc}
%\usepackage[T1]{fontenc} %unnecessary except for maybe below
\usepackage{amsmath}
\usepackage{amsfonts}
\usepackage{amssymb}
\usepackage{wasysym}
\usepackage{mathtools}
%\usepackage{mathabx} %causes \begin document error when loaded
\usepackage{stackengine}
\usepackage{tipa}
\usepackage{multicol}
%\newcommand{\acts{\mathrel{\reflectbox{$\righttoleftarrow$}}}
\setcounter{section}{1}
\author{Todd Hagen}
\title{Notes for \emph{Category Theory for the Sciences}\footnote{David Spivak, MIT Press, 2014}}
\date{}
\begin{document}
\maketitle
\section{The Category of Sets}
A set $X$ is a subset of set $Y$ (i.e. $X \subseteq Y$) if each element in the subset is an element of the set. The null set ($\emptyset$ or $\varnothing$) and the set itself are subsets. 

A function sends each element of a set to an element of another set. Every element of the domain must go to one element of the codomain, but not every element of the codomain has to be ``hit" and codomain elements may be the target of mapping from multiple domain elements. 

A function may be represented by $\rightarrow$ to denote a function from one set to another. The symbol $\mapsto$ (``maps to") tells us where the functions sends a specific element of the domain to the codomain. For example, the squaring function $f: \mathbb{N} \rightarrow \mathbb{N}$ with $5 \mapsto 25$. 

Elements of the codomain (the second set) with a mapping from the domain (the first set) via function $f$  are said to be \textit{in the image} of $f$. 
\begin{equation*}
\text{im}(f) \coloneqq \{y \in Y\ |\ \exists x \in X \text{ such that } f(x) = y\}
\end{equation*} 
The image of $f$ is always a subset of its codomain.
\begin{equation*}
\text{im}(f) \subseteq Y
\end{equation*}

Composable functions share elements of the codomain of the first and the domain of the second. e.g. $f: X \rightarrow Y$ and $g: Y \rightarrow Z$ for elements in im($f$) that are in $g$. Denoted $g \circ f: X \rightarrow Z$, or equivalently in \textit{diagrammatic order} as $f;g: X \rightarrow Z$.

$x \in X$ can be \textit{represented} by the mapping of a function \smiley\ to $X$, denoted $x: \{ \smiley \} \mapsto X$. 

*** The set of functions $X \rightarrow Y$ is denoted by Hom$_{\textbf{Set}}$($X,Y$) ***

\textit{Identity function} on X denoted by id$_{X}$: $X \rightarrow X$, defined as for all $x \in X$ we have id$_{X}$($x$) = $x$.

A function is an \textit{isomorphism} if it is \textit{invertible}, i.e. if there exists an isomorphism $f: X \stackrel{\cong}{\longrightarrow} Y$ there exists an \emph{inverse function} $g: Y \rightarrow X$ such that $g \circ f = $ id$_{X}$ and $f \circ g = $ id$_{Y}$. Sets with an isomorphism between them are called \emph{isomorphic sets}, denoted $X \cong Y$, which is often called a \emph{one-to-one correspondence}\footnote{DNA to RNA trascription is an isomorphism; protein production is a function from 3-nucleotide sequences to eukaryotic amino acids, but it is not an isomorphism ($4^{3}=64$ RNA triplets, for only 21 amino acids.)}. Two sets being isomorphic does not mean they can be treated the same; you need a specific isomorphism in hand, not merely the knowledge that an isomorphic function exists.

Sequence notation: for any set $A$, $f: \underline{n} \rightarrow A$ which can be written as $f = (f(1),f(2),f(3) \ldots f(n))$ for the numeral set of size n, $\underline{n} \coloneqq \{ 1,2,3, \ldots ,n \}$. This just says that a set can be represented as a function from the numeral set to each element of the set.

Cardinality of finite sets: a set A has cardinality n denoted $\left\vert{A}\right\vert = n$, if there exists an isomorphism $A \cong \underline{n}$. Note: a set contains each of its elements only once! Also, the composition of two isomorphisms is an isomorphism. Isomorphic sets have the came cardinality.

\setcounter{subsection}{1}
\subsection{Commutative diagrams}
Diagram of sets and functions between them;  diagram commutes if, for e.g. three functions, two commute to the third---the two paths ``end up" at the same set through compared (commuted) functions.\footnote{e.g. in mechanics, using the Hamiltonian or Langrangian method to predict future states---the predicted future state is the same; the diagram mapping the approaches commutes.}

\subsection{Ologs (ontological logs)}
An olog is a map between math and various conceptual frameworks.
Components of ologs:
\begin{description}
\item[type] An abstract concept (may have multiple parts). A type's label in an olog should be what you call an example of the class.
\begin{itemize}
\item begins with a/an
\item distinction made is recognizable by olog's author
\item instances of which can be documented
\item be the common name that each instance would be called
\item declare all variables, if there are multiple
\end{itemize}

\item[aspect] A way of viewing something, particularly a way of measuring or regarding. In ologs, this means aspects are functions:
\begin{equation*}
\text{aspect: thing to measure } \rightarrow \text{possible outcomes of measurement}
\end{equation*}
Aspects must be functional: often, either switching the direction of the arrow in the olog, or including an intermediary box(type), can turn nonfunctional relationships into proper aspects. Strictly based on functionality; ``unsound" relationships are a product of idiosyncratic worldviews (i.e. views of whoever is constructing the olog) and is not a fault of the aspect in question if it is valid.
\begin{itemize}
\item begins with a verb
\item yields an English sentence when read type1-aspect-type2
\item functional: each instance of source type should give rise to an instance of the target type.
\item usefully describes the functional relationship.
\end{itemize}
Remember: invalid aspects may be converted by flipping the direction of the relationship or introducing a compound type in between the sets in question.

\item[fact] A path equivalence in an olog (i.e. a commutative diagram within the olog); when two paths' functions commute.
\begin{itemize}
\item Can be written as $_{A}[f, g] \simeq _{A}[h, i]$ for starting type $A$ and aspects $f,g,h,i$.
\end{itemize}

\item[image] Recall that the image of a function is the subset of elements in the codomain ``hit" by the function. Image creates new types out of existing aspects; a function is defined first and the resulting type is derived from the function's image.

\end{description}

*Every part (boxes AND arrows) of an olog $\ell$ has an associated English phrase, denoted $\langle\langle \ell \rangle \rangle$.

\section{Fundamental considerations in set}
Products and coproducts; complex limits and colimits; universal properties.
\subsection{Products and coproducts}
\subsubsection{Products}
Definition: the set of ordered pairs $(x,y)$ for $x \in X$ and $y \in Y$, denoted as $X \times Y$. 
\begin{equation*}
X \times Y = \{ (x,y)\ |\ x \in X,\ y \in Y \}
\end{equation*}
One can take the product of more than two sets.\footnote{\ldots which ties into the definition of limits and multidimensional space.
e.g. Three dimensional space as $S \coloneqq \mathbb{R}^{3}$ 
e.g. Spacetime as $S \times T \cong \mathbb{R}^{4}$}

\begin{description}
\item[projection function] Goes from a product to a component set of the product. Denoted by $\pi_{1}: X \times Y \rightarrow X$, $\pi_{2}: X \times Y \rightarrow Y$

\end{description}
The universal property of products: An abstract quality that characterizes a given construction---in products this is the existence of a set $A$ and unique function $A \rightarrow X \times Y$ with functions $f$ and $g$ to $X$ and $Y$ that themselves commute to $X \times Y$. Can be written as
\begin{equation*}
\langle f,g \rangle: A \rightarrow X \times Y, \text{\qquad where \qquad} \langle f,g \rangle(a) = (f(a),g(a))
\end{equation*} 
This function $\langle f,g \rangle$ is said to be \emph{induced} by functions $f$ and $g$. 

\subsubsection{Ologging products}
\begin{itemize}
\item Canonical label for products goes like
\begin{equation*}
\langle \langle c \times d \rangle \rangle \coloneqq \text{``a pair } (x,y) \text{ where } x \text{ is } \langle \langle c  \rangle \rangle \text{ and } y \text{ is } \langle \langle d \rangle \rangle \text{"}
\end{equation*}
\item Projection functions are read as ``yields, as $x$" for the variable $x$ referred to in the projection.
\item Induced function $\langle p, q \rangle : e \rightarrow c \times d$ can be read as 
\begin{equation*}
\text{``yields, insofar as } \langle \langle p  \rangle \rangle \langle \langle c  \rangle \rangle \text{ and } \langle \langle q  \rangle \rangle \langle \langle d  \rangle \rangle \text{,"}
\end{equation*}
\end{itemize}

\subsubsection{Coproducts}
A coproduct is the \emph{disjoint union} of sets, i.e. if an element is in either (or both), it is included in the coproduct. Elements in both sets are each included and labeled with which set they came from.
Coproduct is denoted by $X \sqcup Y$. The coproduct form of a projection function is the \emph{inclusion function},
\begin{equation*}
i_1 : X \rightarrow X \sqcup Y \text{ and } i_2 : X \rightarrow X \sqcup Y
\end{equation*}
Universal property of coproducts: 
For any set $A$ and functions $f: X \rightarrow A$ and $g: A \rightarrow Y$, there exists a unique function $X \sqcup Y \rightarrow A$ induced by $f$ and $g$
\begin{equation*}
\begin{cases}fg\end{cases}: X \sqcup Y \rightarrow A 
\end{equation*}
\begin{equation*}
\begin{cases}fg\end{cases} \circ i_1 = f 
\end{equation*}
\begin{equation*}
\begin{cases}fg\end{cases} \circ i_2 = g
\end{equation*}
The induced function also has to be unique. Assignment as follows guarantees the diagram commutes for the induced function,
\begin{equation*}
\begin{cases}fg\end{cases}(m) = 
\begin{cases}
f(x)& \text{if $m = i_1 x$} \\
g(x)& \text{if $m = i_2 y$}
\end{cases}
\end{equation*}
\emph{A slogan : any time behavior is determined by cases, a coproduct is involved.}

\subsection{Finite limits in Set}
\subsubsection{pullbacks}
\begin{description}
\item[fiber product] is the overlap in images of two functions on the same set. In a diagram, fiber product is denoted by $\lrcorner$
\begin{equation*}
X \times_Z Y \coloneqq \{ (x,y,z)\ |\ f(x) = z = g(y) \}
\end{equation*}
\item[pullback]``a pullback of $X$ and $Y$ over $Z$" --- any set $W$ for which we have an isomorphism $W \rightarrow X \times_Z Y$, i.e. an isomorphism between $W$ and the fiber product of $X$ and $Y$.
Pullbacks may define new ideas from old. Taking the pullback of a diagram defines a new type combining the types (or the functions) that map on to another type.
The universal property of pullbacks: analogous to product/coproduct---there exists a unique function to the fiber product of two sets under a third set. (see diagram 3.12)
Pasting diagrams for pullback:
\begin{equation}
\end{equation}
\item[preimage] denoted $f^{-1}(y)$
\begin{equation*}
f^{-1}(y) \coloneqq \{ x \in X\ |\ f(x) = y \}
\end{equation*}
The preimage is the set of elements of the first set in a function that map to the second set of the function.
For any subset $Y \subseteq Y'$, the \emph{preimage of $Y'$ under $f$} is the set of any elements of $X$ that map into the \emph{subset} of $Y$, $Y'$.
\end{description}
***Remember: you can think of an element $y \in Y$ as a function, ***
\begin{equation*}
y: \{ \smiley \} \rightarrow Y
\end{equation*}

\subsubsection{Spans, experiments, matrices}
\begin{description}
\item[span] a set $R$ together with a functions $f, g$ for which $f: R \rightarrow A$ and $g: R \rightarrow B$. You can think of $A$ and $B$ as the set of observables from a set of experiments $R$ on these variables\footnote{e.g. $T$ is set of possible temps for a gas in a container, $P$ is the set of possible pressures; we have a span $T \leftarrow E \rightarrow P$ for experiment $E$ of changing the temperature and recording both temperature and pressure}.
\emph{composite span} is the fiber product of two sets that span three sets(with one set in common between them). Provides a way to e.g. transparently combine data from multiple sources.
Can be applied to the definition of matrices, as spans of the rows and columns. Composite spans can represent the multiplication of two matrices which share a length of one of their dimensions.
\emph{bitartite graphs} can represent spans; for span $A \stackrel{f}{\leftarrow} R \stackrel{g}{\rightarrow} B$, each element of $A$ drawn as a dot on the left side, each element of $B$ drawn as a dot on the right, with arrows drawn between them to represent $r \in R$ connecting $f(r)$ to $g(r)$.
\end{description}

\subsubsection{Equalizers, terminal objects}
\begin{description}
\item[equalizer] of two functions\footnote{Can depict these functions by $X \underset{g}{\stackrel{f}{\rightrightarrows}} Y$} $f: X \rightarrow Y$ and $g: X \rightarrow Y$ is the subset of elements $x$ in $X$ for which $f(x) = g(x)$. Denoted by $Eq(f,g)$.
Another equalizer diagram example:
\begin{equation*}
\text{\fbox{an input}} \underset{\xrightarrow{\text{according to experiment, yields}}}{\xrightarrow{\text{should, according to theory, yield}}} \fbox{\text{an output}}
\end{equation*}
The equalizer is the set of all inputs for which the theory and the experiment yield the same output.
\item[terminal set] is a set $S$ for which there exists a unique function $X \rightarrow S$ for every set $X$. Has a similar kind of universal property to other constructions in this section (products, pullbacks, and equalizers)---for any other set $S'$ that might fill for $S$, there exists a unique map $S' \rightarrow S$.
\end{description}

\subsection{Finite colimits in Set}
Equivalence relations (denoted $x \sim y$) on $X$ are elements of $R \subseteq X \times X$ (i.e. pairings of the elems in $X$) satisfying the following three properties:
\begin{description}
\item[reflexivity] $(x,x) \in R$;
\item[symmetry] $(x,y) \in R$ iff $(y,x) \in R$;
\item[transitivity] if $(x,z)$ and $(x,y)$ are in $R$, then $(x,z)\in R$.
\end{description}
Equivalence class (denoted by $\sim$) is subset $A \subseteq X$ both
\begin{enumerate}
\item nonempty (i.e. $A \neq \varnothing$)
\item if $x$ in $A$ then $y$ is also in $A$ only if $x \sim y$. 
\end{enumerate}
An equivalence relation on $X$, $\sim$, defines the \emph{quotient of $X$ by $\sim$} as the set of equivalence classes of $\sim$. The quotient function $Q: X \rightarrow X/\sim$ sends $x \in X$ to the equivalence class containing it; we call $x$ a \emph{representative} of its equivalence class $y$ (which is in $X /\sim$).
An example, for $R \subseteq \mathbb{Z} \times \mathbb{Z}$ :

\begin{equation*}
R = \{ (x,y))\ |\ \exists n \in \mathbb{Z}\ \text{such that}\ x + 7n = y \}
\end{equation*}

\begin{description}
\item[reflexive?] Yes, because $x + 7 * (0) = x$.
\item[symmetric?] Yes, $x + 7 * (n) = y$ iff $y + 7 * (-n) = x$
\item[transitive?] Yes, $x + 7 * (n) = y$ and $y + 7 * (m) = z$ together imply $x + 7 * (n+m) = z$.
\end{description}
A \emph{partition of $X$} called $I$, \emph{the set of parts}, has elements $i$ for which the selection of a subset $X_i \subseteq X$ has the following two properties:
\begin{itemize}
\item Every element of $X$ is in some part. i.e. for all $x \in X$, there exists an $i \in I$ such that $x \in X_i$.
\item No element can be found in two different parts
\end{itemize}
Can use partitions to define equivalence classes: $I$ can be the set of equivalence classes in an equiv relation; $X_i$ for each $i$ is the elements of that particular equivalence class.

\emph{Generating equiv relations}:
$X$ is a set, and $R \subseteq X \times X$ any subset.
There exists a relation $S \subseteq X \times X$ such that:
\begin{itemize}
\item $S$ is an equiv relation
\item $R\subseteq S$
\item for any eq relation $S'$ such that $R\subseteq S'$, then $S\subseteq S'$.
\end{itemize}
$S'$ is called the eq relation generated by $R$.

\setcounter{subsubsection}{1}
\subsubsection{Pushouts}
Gluing things together! When you don't want to double count elements as would be done in a coproduct.
For a diagram $Y \stackrel{f}{\longleftarrow} W \stackrel{g}{\longrightarrow} X$,
a \emph{pushout of $X$ and $Y$ over $W$} is any set $Z$ for which we have an isomorphism:
\begin{equation*}
X \sqcup_W Y \stackrel{\cong}{\longrightarrow} Z
\end{equation*}
$X \sqcup_W Y$ is the fiber sum---defined as the quotient of $X\sqcup W\sqcup Y$ by the eq relation $\sim$ generated by $w \sim f(w)$ and $w \sim g(w)$ for all $w \in W$.
The functions from $X$ and $Y$ to $X \sqcup_W Y$ can be called inclusions(more properly: \emph{coprojections}) to be consistent with terminology for coproducts (but these functions are not injective).

Universal property of pushouts:
A pushout $X \sqcup_W Y$ together with inclusions $i_1$ and $i_2$ satisfies the property of:
For any set with functions from $X$ and $Y$, there's a unique arrow 
\begin{equation*}
\left\{\begin{array}{@{} l c @{}}f  g \end{array}\right. : X \sqcup_W Y \rightarrow A
\end{equation*}
that makes everything commute,
\begin{equation*}
f = \left\{\begin{array}{@{} l c @{}}f  g \end{array}\right. \circ i_1
\qquad \text{and} \qquad
g = \left\{\begin{array}{@{} l c @{}}f  g \end{array}\right. \circ i_2
\end{equation*}

\subsubsection{Other finite colimits}
\begin{description}
\item[coequalizer] Given two parallel arrows $X \stackrel{f}\rightarrow Y$ and $X \stackrel{g}\rightarrow Y$, the coequalizer of f and g is the diagram of these parallel arrows with an additional arrow $q: Y \rightarrow Coeq(f,g)$, where we define
\begin{equation}
Coeq(f,g) \coloneqq Y/\sim, \qquad \text{where} \qquad f(x) \sim g(x)\ \text{for all } x\in X
\end{equation}
and $q$ is the quotient function $q: Y \rightarrow Y/\sim$.
\end{description}


\subsection{Other notions in Set}
\subsubsection{Retractions}
Given functions $f: X \rightarrow Y$ and $g: Y \rightarrow X$, and that $g\circ f = \text{id}_X$
\begin{itemize}
\item $f$ is called a \emph{retract section}.
\item $g$ is called a \emph{retract projection}.
\end{itemize}

\subsubsection{Currying}
Taking many inputs in a function can occur one at a time or all at once.\footnote{e.g. A material-specific force-extension curve is a curried form of a function taking a material and an extension and yielding a force.}

\subsubsection{Arithmetic of sets}
Some isomorphisms that exists for any sets $A$, $B$, and $C$:
\begin{multicols}{2}
\begin{itemize}
\item $A + \text{\underline{0}} \cong A$
\item $A + B \cong B + A$
\item $(A + B) + C \cong A + (B + C)$
\item $A \times \underline{0} \cong \underline{0}$
\item $A \times \underline{1} \cong A$
\item $A \times B \cong B \times A$
\item $(A \times B) \times C \cong A \times (B \times C)$
\item $A \times (B + C) \cong (A \times B) + (A \times C)$
\item $A^{\underline{0}} \cong \underline{1}$
\item $A^{\underline{1}} \cong A$
\item $\underline{0}^{A} \cong \underline{0}$
\item $\underline{1}^{A} \cong \underline{1}$
\item $A^{B + C} \cong A^{B} \times A^{C}$
\item $(A^{B})^{C} \cong A^{B \times C}$
\item $(A \times B)^{C} \cong A^{C} \times B^{C}$
\end{itemize}
\end{multicols}
\begin{center}
***\qquad Except for one case: $\underline{0}^{\underline{0}}$\qquad ***
\end{center}
$\underline{0}^{\underline{0}}$ yields conflicting answers for
\begin{enumerate}
\item $A^{\underline{0}} \cong \underline{1}$
\item $\underline{0}^{A} \cong \underline{0}$
\end{enumerate}
The correct answer, based on def's of $\underline{0}$ and $\underline{1}$, is that $\underline{0}^{\underline{0}} = \underline{1}$
\emph{A slogan: any set is isomorphic to any other set with the same number of elements---functions that aren't isomorphisms can't be captured within the framework of numbers. Knowing that there exists an isom between two sets is inferior to having a specific isom in hand---a function can have a pretty arbitrary isom, or a very meaningful one.}

\subsubsection{Subobjects and characteristic features}
\begin{description}
\item[power set]For any set $X$, the set of subsets of $X$. Denoted $\mathbb{P}(X)$.
\item[simplicial complexes]A pair $(V,X)$, where $V$ is a set and $X \subseteq \mathbb{P}(V)$ is subset that is:
\begin{itemize}
\item Downward-closed: for every $u \in X$ and every $u'\subseteq u$, we have $u' \in X$, i.e. anything in the subsets of the elements in $X$ is an elemtent of $X$.
\item Contains all atoms: for every $v \in V$, the singleton set $\{v\}$ is an element of $X$. Atoms are subsets of cardinality 1.
\end{itemize}
Elements of $X$ are called simplices (pl. of simplex).
Set of simplices with cardinality $n + 1$ is denoted by $X_n$, because $X$ is composed of $n$-dimensional elements that together form an $(n+1)$-dimensional ``shape". Each $x \in X_n$ is called an $n$-simplex.
\emph{Since $X$ contains all atoms (singleton sets), we have an isomorphism $X_0 \cong V$. Can call 0-simplices \emph{vertices}, and 1-simplices \emph{edges}.}\footnote{since $X_0 \cong V$, we can denote $(V,X)$ just by $X$.}
Drawing simplicial complex $X$: put all vertices as dots, all $x \in X_1$(denoted as $\Delta^1$, the 1-simplex) as edges (they consist of $(v, v')$ and thus connect vertices), all $y \in X_2$ as filled-in triangles connecting $y = \{w,w',w''\}$.\footnote{Downward-closedness means all edges are also drawn in this step(?).}
\item[subobject classifier] Denoted $\Omega$, defined for \textbf{Set} as the set $\Omega \coloneqq \{True, False\}$ and function $\{\smiley\} \rightarrow \Omega$ sending the unique element to $True$.
A function $X$ to $\Omega = \{True, False\}$ is like ``roll call''--we are interested in the subset that calls $True$.
\item[characteristic function] ``The characteristic function of $A$ in $X$'' is:
\begin{equation*}
\psi (A): X \rightarrow \Omega \qquad \text{by} \qquad \psi (i)(x)=
\begin{cases}
True & \text{if } x \in A \\
False & \text{if } x \notin A
\end{cases}
\end{equation*}
\end{description}

\subsubsection{Surjections, injections}
A definition using elements; more robustly defined using functions:
\begin{center}
$f: X \rightarrow Y$ is \ldots
\end{center}
\begin{description}
\item[injective ($X \hookrightarrow Y$)] if for all $x,x' \in X$ , with $f(x) = f(x')$ we have $x = x'$. 
\quad \emph{In functions}: iff $f$ is a monomorphism.
\begin{description}
\item[monomorphism] if for all sets $A$ and pairs of functions $g,g': A \rightarrow Y, \text{ if } f \circ\ g = f \circ\ g', \text{then } g = g'$.
We can label monomorphisms in diagrams as: ``is''.
\end{description}
\item[surjective ($X \twoheadrightarrow Y$)] if for all $y \in Y$, there is some $x \in X$ s.t. $f(x) = y$.
\quad \emph{In functions}: iff $f$ is a epimorphism.
\begin{description}
\item[epimorphism] if for all sets $B$ and pairs of functions $h,h': Y\rightarrow B,\text{ if } h \circ f = h' \circ f, \text{then } h = h'$.
\end{description}
\item[bijective ($X \stackrel{\cong}{\longrightarrow} Y$)] if $f$ is both injective and surjective.
\quad \emph{In functions}: iff $f$ is both a monomorphism and an epimorphism---i.e. it is an isomorphism.
\end{description}

Fiber product squares link two sets, their fiber product, and the set the fiber product is ``over''. If there exists a monomorphism $g: A \rightarrow Y$, then the left-hand map $g': X \times_Y A \rightarrow X$ is also a monomorphism. 
In the map for this, $g$ is any monomorphism, and $f: X \rightarrow Y$ is any map. 
We can label $\langle\langle X \times_Y A \rangle\rangle \coloneqq ``\langle\langle X \rangle\rangle, \text{which } \langle\langle f \rangle\rangle \langle\langle A \rangle\rangle$''.

\subsubsection{Multisets, relative sets, and set-indexed sets}
Looking at categories other than \textbf{Set}.
\begin{description}
\item[multisets]contain a set of names and their multiplicities (positive finite number of occurrences). Spivak defines a multiset in the following way:
A multiset $X$ is a sequence $X \coloneqq (Oc, N, \pi)$, where $Oc, N$ are the occurrences and names of the multiset, respectively, and $\pi: Oc \rightarrow N$ is surjective (all occurrences are named, so this makes sense) and called the \emph{naming function for $X$}.
\begin{description}
\item[naming function] a function $\pi$ from $Oc$ to $N$, $\pi: Oc \rightarrow N$, in a multiset.
\end{description}
A \emph{mapping from $X$ to $Y$} (for multisets $X$ and $Y$), is a pair of functions mapping $Oc$ and $N$ of $X$ to $Y$.
\item[relative sets] A \emph{relative set over $N$} is a pair ($E,\pi$)
such that $E$ is a set and $\pi: E\rightarrow N$. A mapping of relative sets is $f: (E,\pi)\rightarrow (E', \pi')$ for which $\pi = \pi' \circ f$.
\item[indexed sets] for a set $A$, if we assign each element $a \in A$ a set $S_a$, then $S$ is an $A$-indexed set. Appropriate mappings between indexed sets preserve indices.
\emph{an $A$ indexed set} is a collection of sets $S_a$, one for each $a \in A$, denoted for now as $(S_a)_{a \in A}$. Each $a$ is called an index.
\emph{an $A$ indexed function $f_a$ from $(S_a)_{a \in A}$ to $(S'_a)_{a \in A}$} is a collection of functions $f_a: S_a \rightarrow S'_a$ for each element $a \in A$.
\end{description}

\section{Categories and functors, without admitting it}
Organizing understanding of specific domains, e.g. (all of which are categories (coming in Ch. 5)):
\begin{description}
\item[monoids] organize thoughts about agents acting on objects.
\item[groups] are monoids with agents only acting in reversible ways.
\item[graphs] include nodes and arrows; flows and connections.
\item[databases] are connection patterns for structuring information\footnote{Databases subsume everything else in Ch. 4.}.
\end{description}
The above and \textbf{Set} are all categories---which we can think of as a bunch of objects and a connection pattern between them.
Categories(and Sets) have an interior and exterior view: each example \emph{is} a category, and there is a category \emph{of} that example.
\textbf{This chapter talks about categories without yet generalizing their structure.}
\subsection{Monoids}
A slogan: the performance of a sequence of actions is itself an action.
Monoids and groups are math structures for capturing the agent's perspective---what they can do, what happens when actions are performed in succession. Can be thought of as a set of actions and a formula for how sequences of these actions are themselves considered actions.

\begin{description}
\item[monoid] A sequence $(M,e,\star)$, $M$ is a set, $e$ is an element of $M$ (called the \emph{unit element}), and the \emph{multiplication formula}\footnote{This is ``multiplication'' only in the sense that it is a formula for taking two inputs and returns an output.} $\star: M \times M \rightarrow M$ such that the following \emph{monoid laws} hold\footnote{First form of each is called \emph{infix notation}.}:
\begin{description}
\item[Unit Law 1] $m \star e = m$\quad also written as\quad $\star(m,e) = m$
\item[Unit Law 2] $e \star m = m$\quad also written as\quad $\star(e,m) = m$
\item[Associativity Law] $(m \star n) \star p = m \star (n \star p)$\quad also written as\quad $\star(\star(m,n),p) = \star(m,(\star(n,p))$
\end{description}
\end{description}
Example: additive monoid of addition, denoted usually as $(\mathbb{N}, 0, +)$. The three laws hold and thus we've given $\mathbb{N}$ the structure of a monoid.
It's tempting to treat the set $M$ as the whole monoid if $e$ and $\pi$ are clear from context, but sets may have multiple monoidal structures. We can have an \emph{operation} on a set that is not the formula for any monoid (e.g. exponentiation of the natural numbers --- does not apply to any monoid), or an apparent formula may not even be an operation (e.g. exponentiation on $\mathbb{Z}$).
\begin{description}
\item[trivial monoid] The monoid with only one element, $M = (\{e\},e,\star)$, where $\star: \{e\} \times \{e\} \rightarrow \{e\}$. Sometimes denoted $\underline{1}$.
\end{description}
The order of taking a product doesn't matter in a monoid. A generalization of this is known as the \emph{coherence theorem}\footnote{can be found in Mac Lane's textbook.}.
\begin{description}
\item[list] For a set $X$, a pair $(n,f)$ where the \emph{length of the list} $n \in \mathbb{N}$, and the function $f: \underline{n} \rightarrow X$ where $\underline{n} = \{1,2,\ldots,n\}$.
The \emph{empty list} has $\underline{n} = 0$ and may be denoted [ ].
A \emph{singleton list} for elem $x \in X$ is the list $[x]$.
The \emph{ith entry} of list $L = (n,f)$ for $i \in \mathbb{N}$ and $i \le n$, is the element $f(i) \in X$.
The concatenation of two lists $L$ and $L'$ denoted $L++L'$ is the list $n+n':f++f'$, where $f++f':\underline{n + n'} \rightarrow X$ is given on $1 \le i \le n + n'$ by
\begin{equation*}
(f++f')(i) \coloneqq
\begin{cases}
f(i) & \text{if } 1 \le i \le n, \\
f'(i - n) & \text{if } n+1 \le i \le n + n'
\end{cases}
\end{equation*}
\qquad Concatenating any list $\ell$ with [ ] yields $\ell$.

\item[free monoid] The \emph{free monoid generated by $X$} is
\begin{equation*}
F_X \coloneqq (\text{List}(X),[\ ], ++)
\end{equation*}
Where List($X$) is the list of elements in $X$, [ ] is the empty set, ++ is the concatenation function. $X$ is referred to as the set of \emph{generators} for the monoid $F_X$.

\item[congruence] An equivalence relation $\sim$ on monoid $\mathcal{M} \coloneqq (M, e, \star)$ such that for any $m,m' \in M$ and any $n,n' \in M$, if $m \sim m'$ and $n \sim n'$, then $m \star n \sim m' \sim n'$. The following facts hold for $\mathcal{M}$:
\begin{itemize}
\item The congruence $S$ \emph{generated by relation $R \subseteq M \times M$} is the smallest congruence $S$ containing $R$.
\item If $R = \varnothing$ and generates congruence $\sim$ then there is an isomorphism $M \stackrel{\cong}{\longrightarrow} (M/\sim)$.
\item With $\sim$ as a congruence on $M$, there is a monoid structure $\mathcal{M}/\sim$ on the quotient set $M/\sim$ compatible with $\mathcal{M}$.
\end{itemize}

\item[presented monoid] For finite set $G$ and relation $R \subseteq \text{List}(G) \times \text{List}(G)$, the monoid \emph{presented by generators $G$ and relations $R$} is defined via:

\begin{enumerate}
\item the free monoid $F_G = (\text{List}(G),[ ], ++)$ generated by $G$.
\item $\sim$ denotes the congruence $F_G$ generated by $R$
\item define $\mathcal{M} \coloneqq F_G / \sim$.
\end{enumerate}
In the relation $R$, the elements $r \in R$ are of the form $r = (\ell,  \ell')$ for lists $\ell, \ell' \in \text{List}(G)$.
\emph{A slogan: A presented monoid is a set of buttons to press, and some facts about when different sequences of presses have the same results.}

\item[cyclic monoids] have a presentation involving only one generator.

\item[monoid action] For monoid $(M,e,\star)$ and set $S$, an \emph{action of monoid $M$ on $S$} is a function
\begin{equation*}
\footnote{symbol is in mathabx which I can't load at the moment.} : M \times S \rightarrow S
\end{equation*}
for which the following laws hold:
\begin{itemize}
\item $e  s = s$\quad or in infix \quad $\alpha(e,s) = s$
\item $m (n s) = (m \star n) s$\footnote{this is a ``left action". Right actions have the sides of the equation switched.}\quad or in infix \quad $\alpha(m, \alpha(n,s)) = \alpha(m \star n, s)$
\end{itemize}
Monoid actions as ologs.

\item[finite state machine] denoted by $(\Sigma, S, s_0, \delta, F)$.
\begin{enumerate}
\item $\Sigma$, the input alphabet: a finite nonempty set of symbols.
\item $S$, the state set: finite, nonempty set.
\item $s_0$, the initial state, an element of $S$.
\item $\delta$, the state-transition function; $\delta: \Sigma \times S \rightarrow S$.
\item $F$, the set of final states; $F \subseteq S$.
\end{enumerate}
Giving $\delta$ is equivalent to an action of the free monoid List($\Sigma$) on $S$. \\\\
\emph{A finite state machine is an action of a free monoid on a finite set.}

\item[monoid homomorphism\footnote{For two monoids to be comparable their \emph{sets, unit elements, and multiplication formulas} should be similar; these are the components of a monoid.}] denoted by $f: \mathcal{M} \rightarrow \mathcal{M'}$ for homomorphism $f$ between monoids $\mathcal{M} \coloneqq (M, e, \star)$ and $\mathcal{M'} \coloneqq (M', e', \star')$.
\\\\
The following laws hold for $f$:
\begin{itemize}
\item $f(e) = e'$
\item $f(m_1 \star m_2 = f(m_1) \star' f(m_2),\ \text{for all } m_1, m_2 \in M$
\end{itemize}
The set of homomorphisms between monoids is denoted by Hom$_\textbf{Mon} (\mathcal{M}, \mathcal{M'})$
\end{description}








\subsection{Groups}












\section{Basic category theory}
\section{Fundamental considerations of categories}
\section{Categories at work}


%%%%	below are notes to keep track of throughout reading CT4S	%%%%%
\rule{\textwidth}{0.4pt}
\begin{description}
\item[universal properties] of...
\begin{itemize}
\item Products
\item Coproducts
\item Pullbacks
\item Pushouts
\end{itemize}
\end{description}




\end{document}
